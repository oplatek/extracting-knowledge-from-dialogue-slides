\RequirePackage{silence}
\documentclass[10pt, compress,british,xcolor={svgnames,dvipsnames,x11names},trans]{beamer}

\usepackage[english]{babel}
\usepackage[utf8]{inputenc}
\usepackage{csquotes}
\usepackage{comment}
\usepackage{tikzsymbols}

\def\area#1{{\color{darkgreen}area:\it #1}}
\def\food#1#2{{Dial. state #1: \color{blue}food:\it #2}}
\def\pricerange#1{{\color{orange}pricerange:\it #1}}
\def\sys#1{{\color{purple}System: \it #1}}
\def\usr#1{{\color{brown}User: \it #1}}
\def\api#1{{\color{blue}DB annotations: \it #1}}

% OP - Ondrej Platek inline 
\definecolor{purple}{RGB}{200,0,200}
\def\OP#1{{\color{purple}OP: \it #1}}
\def\OPdel#1{\bgroup\markoverwith{\textcolor{purple}{\rule[0.5ex]{2pt}{1pt}}}\ULon{#1}}


\usepackage{eso-pic}
\beamertemplatenavigationsymbolsempty
\newcommand\AtPagemyUpperLeft[1]{\AtPageLowerLeft{%
\put(\LenToUnit{0.91\paperwidth},\LenToUnit{0.915\paperheight}){#1}}}
\AddToShipoutPictureFG{
  \AtPagemyUpperLeft{{\includegraphics[width=.8cm,keepaspectratio]{logo_ufal_165.png}}}
}%


%%% mtheme customisations
% \usetheme[frametitleformat=regular,progressbar=frametitle,block=fill]{m}
\usetheme[progressbar=frametitle]{m}
\AtBeginSubsection{
\metroset{color/background=dark}
\frame[plain,c]{
  \begin{center}
  \begin{minipage}{25em}
    \usebeamercolor[fg]{section title}
    \usebeamerfont{section title}
    \insertsubsection\\[-1ex]
    \usebeamertemplate*{progress bar in section page}
  \end{minipage}
  \end{center}
}
\metroset{color/background=light}
}
%%%%% end mtheme

\setbeamertemplate{frametitle continuation}[from second]
\setbeamertemplate{bibliography item}[book]

\usetikzlibrary{arrows}
\usetikzlibrary{chains}
\usepackage{tikz-qtree}
\usepackage{multicol}


\usepackage{expex}
%\lingset{glhangindent=2em,glspace=1em,aboveexskip=0pt,belowexskip=0pt,aboveglftskip=-3pt,extraglskip=3pt} %v0.1
%\lingset{exskip=0pt,interpartskip=-3pt,belowpreambleskip=-3pt,belowglpreambleskip=-3pt,aboveglftskip=-3pt,extraglskip=3pt,glhangstyle=none}
\usepackage{relsize}
\usepackage{booktabs,tabularx}
%\usepackage{textcomp}
\usepackage{listings}
\lstset{basicstyle=\ttfamily,breaklines=true,breakatwhitespace=true,
keywordstyle={\color{NavyBlue}\bfseries}, showstringspaces=false,
commentstyle={\color{PaleVioletRed4}},
emphstyle={\color{OliveGreen}\bfseries}
}

\usepackage{algorithmic}
\renewcommand{\algorithmiccomment}[1]{\alert{/* #1 */}}

\usetikzlibrary{shapes.multipart}
\usetikzlibrary{positioning}
\usetikzlibrary{arrows.meta}

\makeatletter
\pgfarrowsdeclare{crow's foot}{crow's foot}
{
  \pgfarrowsleftextend{+-.5\pgflinewidth}%
  \pgfarrowsrightextend{+.5\pgflinewidth}%
}
{
  \pgfutil@tempdima=0.5pt%
  \advance\pgfutil@tempdima by.25\pgflinewidth%
  \pgfsetdash{}{+0pt}%
  \pgfsetmiterjoin%
  \pgfpathmoveto{\pgfqpoint{0pt}{-6\pgfutil@tempdima}}%
  \pgfpathlineto{\pgfqpoint{-6\pgfutil@tempdima}{0pt}}%
  \pgfpathlineto{\pgfqpoint{0pt}{6\pgfutil@tempdima}}%
  \pgfusepathqstroke%
}

\usepackage[os=win]{menukeys}
\usepackage{notoccite}
\usepackage[numbers,sort&compress]{natbib}


\title{{Extracting Knowledge from Dialogue}}
\titlegraphic{logo_ufal_165.png}


\author{Ondřej Plátek \\ \footnotesize{\texttt{oplatek@ufal.mff.cuni.cz}}}
\institute{
Institute of Formal and Applied Linguistics\\
Faculty of Mathematics and Physics\\
Charles University in Prague
}

\begin{document}

\maketitle

\section{Motivation}

\begin{frame}\frametitle{Why end-to-end system?}
    \begin{columns}
        \begin{column}{0.65\textwidth}
            \begin{itemize}
                \item Recurrent neural network composed of sequentially ordered cells
                \item Maintains (encodes) hidden state and emits observation after each turn
                \item Improved with LSTM cells
                \item Suitable for DST task - history is taken into account
            \end{itemize}
        \end{column}
        \begin{column}{0.35\textwidth}
            \includegraphics[width=0.90\textwidth]{encoder.pdf}
        \end{column}
    \end{columns}
\end{frame}

\begin{frame}
    \frametitle{Why learning (explicit knowledge)?}
    \begin{columns}
        \begin{column}{0.65\textwidth}
            \begin{itemize}
                \item Recurrent neural network composed of sequentially ordered cells
                \item Maintains (encodes) hidden state and emits observation after each turn
                \item Improved with LSTM cells
                \item Suitable for DST task - history is taken into account
            \end{itemize}
        \end{column}
        \begin{column}{0.35\textwidth}
            \includegraphics[width=0.90\textwidth]{encoder.pdf}
        \end{column}
    \end{columns}
\end{frame}

% \plain{The End\\Thank you!}

\appendix

\begin{frame}[allowframebreaks]
        \frametitle{References}
        \bibliographystyle{plain}
        \bibliography{literature.bib}
\end{frame}

\end{document}
